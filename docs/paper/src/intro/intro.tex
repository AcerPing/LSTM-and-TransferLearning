\chapter{序論}

\section{背景}
% \memo{創薬の一般的な状況}
創薬の現場において、構造の探索は一般に化学者・薬学者の知見に基づいて行われている。
一方で実験には一定の時間およびコストがかかることから、実験サイクルの最適化が求められている。

% \memo{QSARの流れ}
実際に実験を行うことなく活性を予測する手法として、
化合物の特徴を表す記述子と活性値との関係を統計的にモデリングする
\term{定量的構造活性相関}[Quantitative Structure-Activity Relationships][QSAR]
が存在する。一般的なQSARは以下のステップに基づいて行われる。
\begin{enumerate}
\item 化学構造から特徴を抽出した記述子を計算
\item 記述子と活性値との間にモデルを構築
\end{enumerate}
この方法に基づいて構築したモデルを利用すると、化学構造から活性値を予測することができる。

% \memo{inverse-QSARの難しさ・先行研究}
一方、活性値から対応する化学構造を得る枠組みをinverse-QSARという。
\figref{fig:qsar_scheme}にQSARとinverse-QSARの流れを示す。
望ましい活性を示すような化学構造を得るinverse-QSARを考えた際には2つの問題点が存在する。
1つ目は一般的なQSARモデルは複雑な非線形関数であるため、活性値から記述子を得るような逆関数が明示的に得られず、逆解析が困難なことである。
もう1つは記述子が得られた場合でもその記述子を持つような化学構造を組み立てるのが困難であるということである。
これらの問題を解決するため、宮尾らは\term{Gaussian Mixture Model}[GMM]を利用して逆関数を明示的に得る手法\cite{Miyao}や
Differencial Evolutionを用いて非線形モデルに対して望ましい記述子を得る手法\cite{Miyao2017}を提案している。
\begin{figure}[tbp]
    \centering
    \includegraphics[width=0.8\columnwidth]{../resource/qsar_scheme.png}
    \caption{QSARとinverse-QSAR} \label{fig:qsar_scheme}
\end{figure}

% \memo{深層生成モデルアプローチ}
近年盛んに研究されているアプローチは、深層生成モデル(後述)を用い、記述子を計算することなく望ましい活性を持つ化学構造を直に得るという手法である。
G\'{o}mez-Bombarelliらは\term{Variational Autoencoder}[VAE] \cite{Kingma2014, Doersch2016}を用いて、
化学構造の線形表記の1種であるSMILESから連続値の潜在表現を得る手法を提案した\cite{Gomez-Bombarelli2016}。
潜在変数空間中の1点を指定すればそこから直にSMILES表記された化学構造を得ることができることを示したほか、
いくつかの物性について潜在変数空間を探索することで最適化が可能であることを確認している。
BlaschkeらはVAEに加え、\term{Adversarial Autoencoders}[AAE] \cite{aae}というモデルを採用した。
深層生成モデルを学習させた後に\term{Dopamine Receptor D2}[DRD2]のデータセットに対して実際に構造最適化を行い、
活性が高いと思われる構造を取得することに成功している\cite{Blaschke2018}。

\section{本研究の目的と方針}

本研究では、深層生成モデルを利用したいずれの先行研究も検証を大規模データで行っていることに着目した。
G\'{o}mez-Bombarelliら\cite{Gomez-Bombarelli2016}はZINCデータベースから抽出した$250,000$件のデータを、
Blaschkeら\cite{Blaschke2018}はExCAPE-DBから抽出した$350,422$件のデータを用いている。
これらはいずれも教師付き(物性・活性値の測定されている)データである。
一方でタンパク質を特定した場合、教師付きデータは数十件から数百件程度しか存在しないことも多い。
教師付きデータが少ない場合は深層生成モデルはうまく学習することができないため、
少数データに対して利用可能な深層生成モデルは未だ提案されていない。
そこで、本研究では教師付きデータが少ない場合でも良好な構造提案が行える手法の提案を目的とした。

方針としては、VAEとAAEを半教師付き学習的に利用した。
半教師付き学習とは、多数の教師なしデータを利用してデータの分布を学習し、
学習した分布を用いて教師との関係を学習する方式である。
半教師付き学習を利用すると、手元に実験値が存在するデータがわずかであっても、
Web上で公開されているオープンデータの化学構造と組み合わせて学習を行うことができる。

また、化学データに限らなければ\term{Generative Adversarial Positive-Unlabeled Learning}[PUGAN] \cite{pugan}という半教師付き学習の手法が存在する。
これも化学データに適用し、結果を検証した。